\section{\sc Research Interests}
% I am a third year Ph.D. student at Politecnico di Milano, in the Department of Chemistry,
% Materials and Chemical Engineering Giulio Natta, within the CRECK modeling laboratory,
% under the supervision of Prof. Alessandro Stagni. My research activity focuses on the
% application of data-driven methods to promote the development of chemical kinetics
% mechanisms for the prediction of combustion and pyrolysis of complex fuels. Overall the
% main interests are:
%
% \begin{itemize}
%    \item Developing and maintaining an automatic framework for the
%       collection and analysis of scientific data and kinetic models
%       (\href{https://sciexpem.polimi.it/}{SciExpeM}).
%
%    \item Build reliable pipelines for developing and validating detailed kinetic
%       mechanisms against large numbers of experimental measurements using data science
%       techniques.
%
%    \item Developing automatic routine for optimizing complex kinetic schemes empwering physics-based approaches.
%
%    \item Apply different algorithms for the inverse modeling of high dimensional problems.
% \end{itemize}

I am a third-year Ph.D. candidate enrolled at Politecnico di Milano, affiliated with the
Department of Chemistry, Materials, and Chemical Engineering Giulio Natta. I am actively
engaged in research within the CRECK modeling laboratory, under the guidance of Professor
Alessandro Stagni. My doctoral research is primarily centered on leveraging data-driven
methodologies to advance the development of chemical kinetics mechanisms aimed at
predicting combustion and pyrolysis behaviors of complex fuels. My academic pursuits
revolve around several key areas of interest:

\begin{itemize}
   \item Establishing and managing an automated framework for the systematic collection
      and analysis of scientific data and kinetic models
      (\href{https://sciexpem.polimi.it}{SciExpeM}).

   \item Constructing robust pipelines for the creation and validation of detailed kinetic
      mechanisms, which are rigorously benchmarked against extensive experimental datasets
      employing data science methodologies.

   \item Designing automated routines to streamline the optimization of complex kinetic
      schemes, thereby enhancing the efficacy of physics-based approaches.

   \item Implementing various algorithms for conducting inverse modeling of
      high-dimensional problems, with a focus on enhancing predictive capabilities in
      complex systems.
\end{itemize}
